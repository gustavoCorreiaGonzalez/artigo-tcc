\chapter{Introdução}

	O ensino de linguagens de programação visa propiciar aos alunos o desenvolvimento de um conjunto de competências necessárias para conceber programas e sistemas computacionais capazes de resolver problemas reais. O insucesso na aprovação dos estudantes em disciplinas de programação, é um tema que tem sido alvo de alguns estudos \cite{bosse2015reprovaccoes, Cukierman:2015:PSU:2729094.2742623}. Este estudo pretende realizar a implantação de um mecanismo de dicas para resolução de exercícios afim de descobrir se o desempenho e interesse dos alunos melhoram na disciplina. 
	
	Com a necessidade de prover um suporte personalizado aos alunos, \citeonline{Elkherj:2014:SSR:2556325.2567864} criaram um sistema de dicas que permite ao professor enviar uma dica personalizada em tempo real após verificar que o aluno realizou várias tentativas para resolver um exercício. Entretanto, com o aumento do número de alunos utilizando o sistema, os professores não conseguem oferecer suporte a todos eles. Assim, a utilização desse tipo de abordagem na construção do sistema de dicas apresenta limitações em relação a quantidade de alunos realizando os exercícios.
	
	O método que será aplicado para a implementação do sistema é o \foreign{learnersourcing} que gerencia as atividades dos alunos através de uma interface que coleta os dados de aprendizagem dos usuários, suas avaliações de explicações e elícita a geração de novas explicações para futuros alunos.

	Nós utilizamos a abordagem do \foreign{learnersourcing} apresentada por \citeonline{Glassman:2016:LPH:2818048.2820011} com o intuito de que os alunos, através de sua própria experiência resolvendo exercícios, possam criar dicas úteis através de suas implementações, gerando sugestões para colegas com base em sua própria experiência.