\chapter{Introdução}

O ensino de linguagens de programação tem o propósito de conseguir desenvolver nos aluno um conjunto de competências necessárias para conceber programas e sistemas computacionais capazes de resolver problemas reais. O insucesso na aprovação dos estudantes em disciplinas de programação, é um tema que tem sido alvo de alguns estudos \cite{bosse2015reprovaccoes}, \cite{Cukierman:2015:PSU:2729094.2742623}.

O problema do insucesso é evidenciado por \cite{lahtinen2005study} realizaram uma pesquisa com 559 alunos e 34 professores de diferentes universidades para estudar as dificuldades na aprendizagem de programação, como resultado obtido foi percebido que as questões mais difíceis na programação são: a compreensão de como projetas um programa para resolver uma tarefa determinada, dividir as funcionalidades em procedimentos e encontrar erros de seus próprios programas. Estas são as capacidades que o aluno deve obter para entender as maiores entidades do programa em vez de apenas alguns detalhes sobre eles. Já os conceitos de programação mais difíceis foram recursão, ponteiros e referências, tipos de dados abstratos, manipulação de erro. Os professores apontaram como sendo os conteúdos mais difíceis os mesmos que os alunos.

O objetivo desse trabalho é criar um software de código aberto para auxiliar na aprendizagem de programação, implementando um sistema colaborativo de dicas escritas pelos próprios usuários. Para investigar se do uso do mecanismo de dicas personalizadas é capaz de melhorar o rendimentos dos alunos nos exercícios de estrutura de condição e laço de repetição.

No software serão implementadas funcionalidades que permitam o usuário resolver exercícios em qualquer linguagem de programação, fornecer dicas para a solução de exercícios, um diário para relatar as dificuldades enfrentadas durante a execução dos exercícios.

Para realizar a validação da ferramenta será realizado um estudo controlado na Universidade Tecnológica Federal do Paraná com três grupos de estudantes escolhidos aleatóriamente de forma homogênea em relação ao nível de conhecimento na programação, um grupo utilizará o software de dicas normal, outro irá dispor de dicas personalizado e o último utilizará o software sem o sistema de dicas personalizadas.

Os próximos capítulos estão organizados em três partes. O objetivo do capítulo de referencial teórico é apresentar e discutir os principais conceitos que envolvem este trabalho. Este  procedimento é importante, pois discutiremos os pontos de vista de diversos autores, assim como diversas abordagens alternativas. O objetivo do capítulo de proposta é apresentar a metodologia que será utilizada para desenvolver o estudo e o software. O objetivo do capítulo de resultados preliminares é apresentar os avanços que o estudo obteve durante o seu desenvolvimento.




