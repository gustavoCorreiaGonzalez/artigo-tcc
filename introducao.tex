\chapter{Introdução}

O ensino de linguagens de programação tem o propósito de conseguir desenvolver nos alunos um conjunto de competências necessárias para conceber programas e sistemas computacionais capazes de resolver problemas reais. O insucesso na aprovação dos estudantes em disciplinas de programação, é um tema que tem sido alvo de alguns estudos \cite{bosse2015reprovaccoes, Cukierman:2015:PSU:2729094.2742623}.

O problema do insucesso é evidenciado por \citeonline{lahtinen2005study}, que realizaram uma pesquisa com diferentes universidades para estudar as dificuldades na aprendizagem de programação. Como resultado foi percebido que as questões mais difíceis na programação são: a compreensão de como projetar um programa para resolver uma tarefa determinada, dividir as funcionalidades em procedimentos e encontrar erros de seus próprios programas. Estas são as capacidades que os alunos devem obter para entender as maiores entidades do programa em vez de apenas alguns detalhes sobre eles.

Estas dificuldades contribuem para a desistência dos alunos nas disciplinas de programação. Entretanto, os pesquisadores estão utilizando várias metodologias e \foreign{softwares} para minimizar esse problema. Este estudo pretende realizar a implantação de um mecanismo de dicas para resolução de exercícios afim de descobrir se o desempenho e interesse dos alunos melhoram na disciplina. 

Com a necessidade de prover um suporte personalizado aos alunos, \citeonline{Elkherj:2014:SSR:2556325.2567864} criaram um sistema de dicas que permita o professor enviar uma dica personalizada em tempo real após verificar que o aluno realizou várias tentativas para resolver o exercício. Entretanto, com o aumento do número de alunos utilizando o sistema, os professores não conseguem oferecer suporte a todos os alunos. Assim, a utilização desse tipo de abordagem na construção do sistema de dicas apresenta limitações em relação a quantidade de alunos realizando os exercícios.


% TODO: Melhorar esse parágrafo
O método que será aplicado para a implementação do sistema é o \foreign{learnersourcing} que gerencia as atividades dos alunos através de uma interface que coleta os dados de aprendizagem dos alunos e suas avaliações de explicações e elícita a geração de novas explicações de futuros alunos.

Nós utilizamos a abordagem do \foreign{learnersourcing} apresentada por \citeonline{Glassman:2016:LPH:2818048.2820011} com o intuito de que os alunos, através de sua própria experiência resolvendo exercícios, podem criar dicas úteis através de suas implementações. Estes alunos podem então gerar sugestões para colegas com base em sua própria experiência.

O objetivo desse trabalho é criar um software de código aberto para auxiliar na aprendizagem de conceitos básicos de programação, implementando um sistema colaborativo de dicas escritas pelos próprios usuários. Para investigar se o uso do mecanismo de dicas personalizadas é capaz de melhorar o rendimento dos alunos nos exercícios de estrutura de condição e laço de repetição, será aplicado um experimento controlado com três grupos distintos de alunos. O primeiro grupo utilizará o mecanismo de dicas que consiste em prover dicas de acordo com o exercício que o aluno está realizando. O segundo grupo utilizará o mecanismo de dicas personalizado que disponibiliza dicas de acordo com o exercício e o erro cometido na submissão. Por fim, o terceiro grupo utilizará o sistema sem o mecanismo de dicas.

No software serão implementadas funcionalidades que permitam o usuário resolver exercícios nas linguagens de programações C, C++ e Java, fornece dicas para a solução de exercícios, um diário para relatar as dificuldades enfrentadas durante a execução dos exercícios. Estas informações são úteis e podem ser personalizar e melhorar o ensino de programação.

Para realizar a validação da ferramenta será realizado um estudo controlado na UTFPR com três grupos de estudantes escolhidos aleatoriamente de forma homogênea em relação ao nível de conhecimento em programação. Com o experimento será possível avaliar se o grupo de alunos com dicas tem melhor desempenho que alunos sem o suporte do mecanismo. O desempenho será medido por: tempo, número de tentativas até a solução correta, qualidade do código gerado medida através da complexidade ciclomática do código, número de linhas entre as tentativas e tamanho da solução. Por fim, será avaliado se a qualidade das dicas está diretamente relacionada com o nível de conhecimento do aluno, e se as dicas de alunos experientes são melhores ou não em relação as dicas dos alunos com menos experiência.

Os próximos capítulos estão organizados em três partes. No capítulo de referencial teórico será apresentado e discutido os principais conceitos que envolvem o estudo. Este procedimento é importante, pois discutiremos os pontos de vista de diversos autores, assim como diversas abordagens alternativas. O objetivo do capítulo de proposta é apresentar a metodologia que será utilizada para desenvolver o estudo e o software. Por fim, o capítulo do cronograma apresentará as atividades que serão realizadas após o término da escrita do trabalho de conclusão de curso.