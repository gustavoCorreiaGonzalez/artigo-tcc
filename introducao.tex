\chapter{Introdução}

- terceira pessoa do singular - sujeito oculto

O ensino de linguagens de programação tem o propósito de conseguir desenvolver nos aluno um conjunto de competências necessárias para conceber programas e sistemas computacionais capazes de resolver problemas reais. O insucesso na aprovação dos estudantes em disciplinas de programação, é um tema que tem sido alvo de alguns estudos \cite{bosse2015reprovaccoes}, \cite{Cukierman:2015:PSU:2729094.2742623}.

O problema do insucesso é evidenciado por \citeonline{lahtinen2005study} realizaram uma pesquisa com 559 alunos e 34 professores de diferentes universidades para estudar as dificuldades na aprendizagem de programação. Como resultado foi percebido que as questões mais difíceis na programação são: a compreensão de como projetar um programa para resolver uma tarefa determinada, dividir as funcionalidades em procedimentos e encontrar erros de seus próprios programas. Estas são as capacidades que o aluno deve obter para entender as maiores entidades do programa em vez de apenas alguns detalhes sobre eles. Já os conceitos de programação mais difíceis foram recursão, ponteiros e referências, tipos de dados abstratos, manipulação de erro.

% TODO: Colocar um parágragro para dizer que o sistema de dicas pode ser uma possível solução!!

Com a necessidade de prover um suporte personalizado aos alunos, \citeonline{Elkherj:2014:SSR:2556325.2567864} criaram um sistema de dicas que permite o professore enviar uma dica personalizada em tempo real após verificar que o aluno realizou várias tentativas para resolver o exercício. Entretanto, com o aumento do número de alunos utilizando o sistema, os professores não conseguem oferecem um suporte a todos os alunos, assim utilização desse tipo de abordagem na construção do sistema de dicas apresenta limitações em relação a quantidade de alunos realizando os exercícios.

Nós utilizamos a abordagem do \foreign{learnersourcing} com o intuito de que os alunos, através de sua própria experiência resolvendo exercícios, podem criar dicas úteis através de suas implementações ou \foreign{bugs} que resolvem. Estes alunos podem então gerar sugestões para colegas com base em sua própria perícia. Mostraremos que, dada essa escolha de \foreign{design}, os alunos podem criar dicas que podem até mesmo substituir a assistência personalizada dos professores, quando essa assistência não está disponível.



% TODO: Ver com o Igor para melhorar esse parágrafo
O objetivo desse trabalho é criar um software de código aberto para auxiliar na aprendizagem de conceitos básicos de programação, implementando um sistema colaborativo de dicas escritas pelos próprios usuários. Para investigar se o uso do mecanismo de dicas personalizadas é capaz de melhorar o rendimentos dos alunos nos exercícios de estrutura de condição e laço de repetição, nós iremos aplicar um experimento controlado com três grupos distintos de alunos sendo eles:

\begin{itemize}
	\item \textbf{Primeiro Grupo}: Este grupo utilizará o mecanismo de dicas normal que consiste em prover dicas de acordo com o exercício que o aluno está realizado.
	
	\item \textbf{Segundo Grupo}: Este grupo utilizará o mecanismo de dicas personalizado que disponibiliza dicas de acordo com o exercício e o erro cometido na submissão.
	
	\item \textbf{Terceiro Grupo}: Este grupo utilizará o sistema sem o mecanismo de dicas.
	
\end{itemize}

No software serão implementadas funcionalidades que permitam o usuário resolver exercícios nas linguagens de programações C, C++ e Java, fornecer dicas para a solução de exercícios, um diário para relatar as dificuldades enfrentadas durante a execução dos exercícios. Estas informações são utéis e podem ser personalizar e melhorar o ensino de programação.

% TODO: Conversar com o Igor para melhorar o paragráfo
Para realizar a validação da ferramenta será realizado um estudo controlado na Universidade Tecnológica Federal do Paraná com três grupos de estudantes escolhidos aleatóriamente de forma homogênea em relação ao nível de conhecimento na programação. Com o experimento será possível avaliar se o grupo de alunos com dicas tem melhor desempenho que alunos sem o suporte do mecanismo. O desempenho será medido por: tempo, número de tentativas até a solução correta, qualidade do código gerado medida complexidade ciclimática do código, número de linhas entre as tentativas e tamanho da solução. Por fim também serão avaliadas se a qualidade das dicas são diretamente influenciadas com o nível de conhecimento do aluno, e se as dicas de alunos são melhores ou não em relação as dicas dos professores.

Os próximos capítulos estão organizados em três partes. O objetivo do capítulo de referencial teórico é apresentar e discutir os principais conceitos que envolvem este trabalho. Este  procedimento é importante, pois discutiremos os pontos de vista de diversos autores, assim como diversas abordagens alternativas. O objetivo do capítulo de proposta é apresentar a metodologia que será utilizada para desenvolver o estudo e o software. O objetivo do capítulo de resultados preliminares é apresentar os avanços que o estudo obteve durante o seu desenvolvimento.