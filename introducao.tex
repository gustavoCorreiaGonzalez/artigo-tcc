\chapter{Intrução}

O insucesso na aprovação dos estudantes em disciplinas de programação, é um tema que tem sido alvo de muitos estudos \cite{bosse2015reprovaccoes}, sendo que o ensino de linguagens de programação tem o propósito de conseguir desenvolver nos aluno um conjunto de competências necessárias para conceber programas e sistemas computacionais capazes de resolver problemas reais. 

Porém, existe uma grande dificuldade apresentada por parte dos alunos, \cite{lahtinen2005study} realizaram uma pesquisa com 559 alunos e 34 professores de diferentes universidades para estudar as dificuldades na aprendizagem de programação, como resultado obtido foi percebido que as questões mais difíceis na programação são: a compreensão de como projetas um programa para resolver uma tarefa determinada, dividir as funcionalidades em procedimentos e encontrar erros de seus próprios programas. Estas são as capacidades que o aluno deve obter para entender as maiores entidades do programa em vez de apenas alguns detalhes sobre eles. Já os conceitos de programação mais difíceis foram recursão, ponteiros e referências, tipos de dados abstratos, manipulação de erro. Os professores apontaram como sendo os conteúdos mais difíceis os mesmos que os alunos.

Nosso estudo tem como objetivo criar um software de código aberto para auxiliar na aprendizagem de programação, utilizando um sistema colaborativo de dicas escritas pelos próprios usuários. 

Serão implementadas funcionalidades que permitam o usuário resolver exercícios em qualquer linguagem de programação, fornecer dicas para a solução do exercício, um diário para relatar as dificuldades enfrentadas durante a execução dos exercícios e um sistema de \textit{gamification} com relação ao rendimento do usuário.

Para realizar a validação da ferramenta será realizado um estudo com três grupos de estudantes, um com o sistema de dicas normal, outro com o sistema de dicas personalizado e um sem dicas.




