\chapter{Proposta}

Neste capítulo apresenta o método utilizado para a elaboração deste trabalho. A Figura \ref{figura:visaometodo} ilustra cada etapa a ser realizada para o desenvolvimento da abordagem proposta.

\begin{figure}[h]
	\captionsetup{justification=centering}
	\includegraphics[width=\linewidth]{visaogeraldometodoproposto.png}
	\caption{Visão geral do método proposto.}
	\label{figura:visaometodo}
\end{figure}

Para avaliar o impacto do uso do mecanismo de dicas personalizadas no ensino de conceitos básicos de programação, primeiramente será realizado a implementação do sistema de dicas utilizando o \foreign{framework} de desenvolvimento Laravel na versão 5.1. 

%Será realizado a implementação do sistema de dicas utilizando o \foreign{framework} de desenvolvimento Laravel na versão 5.1. A partir do sistema desenvolvido, será realizado a construção da base de dados de exercícios do qual constituirá de exercícios referentes a estrutura condicional e laço de repetição. Na etapa seguinte a partir do banco de dados de exercícios formaremos o bando de dados de dicas, para realizar essa etapa será necessário a colaboração de alguns voluntários para utilizar o sistema de dicas e realizar os exercícios e fornecer dicas para o banco de dicas. Em seguida, será realizado um teste do sistema antes da realização do estudo prártico. Por fim, será realizado a etapa de validação que constituirá de três estudos isolados com volutários que utilização o sistema de dicas. A seguir, todas as etapas serão descritas em suas respectivas seções.

\section{Implementação do Mecanismo de Dicas}

Essa seção, descreve o sistema de dicas a ser desenvolvido. O processo de realização de exercícios, que é onde o sistema deve atuar, pode ser dividi do em quatro fases: cadastro do usuário, realização de exercícios, utilização de dicas (providas por usuários do sistema) e validação do exercício. A descrição apresentada será usada para a modelagem do sistema.

utilizar o asth pq esse ta uma merda!!!

\begin{figure}[h]
	\captionsetup{justification=centering}
	\includegraphics[width=\linewidth]{sistemadicas.png}
	\caption{Visão geral do sistema de dicas.}
	\label{figura:sistemadicas}
\end{figure}

\subsection{Cadastro do Usuário}

Quando um usuário desejar realizar o cadastro no nosso sistema, ele poderá escolher entre a opção de aluno ou professor. O aluno, realiza exercícios disponíveis no banco de dados do sistema, preenche um diário não obrigatório para cada exercício, cria uma dica para cada exercício e consulta seu perfil com seus dados pessoais e um relatório de suas submissões de exercícios. O professor, além de realizar as mesmas atividades do aluno, também poderá criar salas com exercícios pré-definidos convidando alunos para fazer parte dela e submeter exercícios para o sistema.

\subsection{Cadastro de Exercícios}

Essa funcionalidade do sistema só poderá ser acessada se o usuário for um professor. Sendo assim, o professor terá que cadastrar um enunciado, qual linguagem é para ser utilizada na resolução, o nivel (fácil, médio ou difícil), o tipo do exercício (condicional, laço de repetição, função, etc), a resposta que poderá ser escrita em um campo ou um \foreign{upload} do arquivo com as respostas. Cada exercício cadastrado será agregado a uma lista de exercícios que o professor irá criar antes de cadastrar os exercícios.


% TODO: Colocar formato de exemplo de uma resposta

\subsection{Cadastro de Turma}

O professor poderá realizar o cadastro de uma classe para adicionar seus alunos e passar as listas de exercícios e acompanhar o rendimento de cada aluno. Para cadastrar uma turma é necessário um nome da turma e um professor, com isso o professor poderá adicionar os alunos desejados.

\subsection{Realização de Exercícios}

Após a conclusão do cadastro de usuário, o aluno ou professor terão acesso as funcionalidades especificas de cada tipo de usuário. Dessa forma, o usuário poderá resolver uma lista de exercícios de duas formas: estando vinculado a uma turma ou individualmente. Caso o usuário esteja vinculado a uma turma, a lista de exercícios que ele poderá resolver será a que o professor responsável pela turma criou. Caso o usuário não faça parte de uma turma, ele terá que pesquisar por uma lista de exercícios informando qual assunto, estrutura de condição ou laço de repetição, ele deseja realizar. 

Depois que o usuário encontrar uma lista de exercícios, o sistema apresentaŕa os exercícios à serem resolvidos na ordem em que foram adicionados na lista. Assim o usuário poderá trabalhar em sua solução e submeter ao sistema para correção utilizando a linguagem de programação escolhida pelo professor na criação da lista de exercícios, essa resolução poderá estar correta ou não. Se estiver correta, o sistema apresentará uma mensagem de parabenização e redirecionará o usuário para o próximo exercício. Caso a resolução estiver errada o sistema retornará qual o erro ocorrido na compilação, assim o usuário poderá realizar outras submissões até que consiga obter sucesso na resolução ou requisitar uma dica para o sistema para auxiliá-lo na contrução da resposta. 

Cada submissão, tanto correta ou incorreta, é gravada no banco de dados na forma de \foreign{log}, sendo salvo os dados: a usuário que realizou o exercício, todas as submissões, os erros cometidos, o tempo demorado para obter sucesso no exercício, (ver a onde foi marcado no papel).


\subsubsection{Utilização das Dicas}

O usuário apenas poderá consultar uma dica caso o exercício enviado para validação não estaja correto e ele escolha a opção de disponibilização de dicas, assim ele terá direito a três dicas selecionadas aleatoriamente pelo mecanismo. O usuário irá escolher consultar uma das três dicas e após a consulta ele poderá realizar uma nova submissão do exercício.

\subsection{Cadastro de Dica}

O cadastro de uma dica pode ser realizado quando o usuário submete um exercício para validação e obtêm exito, assim o sistema apresentará uma tela perguntando se o usuário deseja contribuir com uma dica, caso aceite, o sistema irá redirecionar o usuário para a tela onde será realizado o cadastro da dica, onde ele irá descrever em um campo a dica que deseja compartilhar. Também será apresentada a solução do mesmo exercício de outro usuário para que seja escrito uma dica. 

\subsection{Cadastro do Diário}

No processo de execução de um exercício o usuário poderá preencher um diário relatando suas experiências, essa funcionalidade será apresentada aos usuário caso o professor deseja receber o preenchimento do diário dos exercícios da lista. 

Caso o professor pretende receber os diários dos exercícios da lista de sua turma, o usuário após cada submissão da resolução no sistema produzirá um diário que será informado os dados referentes a experiência de realizar o exercício. 

\subsection{Boca}

Para realizar a compilação dos exercícios no sistema de dicas, será utilizado funções do \foreign{software} BOCA que é escrito predominantemente em PHP, o que permitiria sua execução em qualquer plataforma compatível com tal linguagem. No entanto, o sistema de julgamento depende de scripts Bash e de funcionalidades específicas do sistema operacional Linux (isolamento de recursos/jail). 

\section{Formato do Estudo}

Esta seção apresentará o formato do estudo, descrevendo como serão realizados o teste do sistema, a criação do banco de exercícios e dicas, como serão respondidas as questões de pesquisa


colocar figura com o workflow do trabalho
ex: 3 grupos -> realizam cadastro -> realizam exercício -> etc

\subsection{Teste do sistema}

- descrever como será realizado o teste do sistema

\subsection{Banco de Exercícios}

O banco de dados de exercícios será criado a partir de exercícios selecionados para diferir dos exercícios realizados na matéria de Algoritmos oferecida pelo curso, a seleção terá a finalidade de previnir que o voluntário do estudo de validação do sistema já tenha realizado o exercício. Os exercícios serão divididos em três grupos, o primeiro abordará conceitos básicos da linguagem de programação como entrada e saída de dados, declaração de varíavies e constantes, operações e funções matemáticas. O segundo grupo refere-se a estruturas condicionais e o terceiro grupo trata de estruturas de repetição.

\subsection{Banco de Dicas}

O banco de dicas será provido após a conclusão do banco de exercícios, assim realizaremos uma convocação por formulário \textit{online} aos estudantes da Universidade Tecnológica Federal do Paraná do curso de Bacharelado em Ciência da Computação para se voluntariarem a utilizar o sistema de dicas por um determinado período em um dia que será estipulado no formulário. Os voluntários utilizaram o sistema de dicas em um ambiente controlado sendo esse um laboratório da Universidade Tecnológica Federal do Paraná com um professor ou alunos monitorando as atividades e tirando as possíveis dúvidas dos participantes. Cada participante terá que realizar todos os exercícios cadastrados na base de dados e cadastrar dicas para cada um.

- colocar o workflow de como será realizado o estudo

- pensar em colocar uma seção das qustoes de pesquisa