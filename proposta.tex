\chapter{Proposta}

Neste capítulo apresenta o método utilizado para a elaboração deste trabalho. A Figura \ref{figura:visaometodo} ilustra cada etapa a ser realizada para o desenvolvimento da abordagem proposta.

\begin{figure}[h]
	\captionsetup{justification=centering}
	\includegraphics[width=\linewidth]{visaogeraldometodoproposto.png}
	\caption{Visão geral do método proposto.}
	\label{figura:visaometodo}
\end{figure}

Para avaliar o impacto do uso do mecanismo de dicas personalizadas no ensino de conceitos básicos de programação, primeiramente será realizado a implementação do sistema de dicas utilizando o \foreign{framework} de desenvolvimento Laravel na versão 5.1. 

\section{Implementação do Mecanismo de Dicas}

Esta seção, descreve o sistema de dicas a ser desenvolvido. O processo de realização de exercícios, que é onde o sistema deve atuar, pode ser dividi do em quatro fases: cadastro do usuário, realização de exercícios, utilização de dicas (providas por usuários do sistema) e validação do exercício. A descrição apresentada será usada para a modelagem do sistema.

\begin{figure}[h]
	\captionsetup{justification=centering}
	\includegraphics[width=\linewidth]{sistemadicas.png}
	\caption{Visão geral do sistema de dicas.}
	\label{figura:sistemadicas}
\end{figure}

\subsection{Cadastro do Usuário}

Quando um usuário desejar realizar o cadastro no nosso sistema, ele poderá escolher entre a opção de aluno ou professor. O aluno, realiza exercícios disponíveis no banco de dados do sistema, preenche um diário não obrigatório para cada exercício, cria uma dica para cada exercício e consulta seu perfil com seus dados pessoais e um relatório de suas submissões de exercícios. O professor, além de realizar as mesmas atividades do aluno, também poderá criar salas com exercícios pré-definidos convidando alunos para fazer parte dela e submeter exercícios para o sistema.

\subsection{Cadastro de Exercícios}

Essa funcionalidade do sistema só poderá ser acessada se o usuário for um professor. Sendo assim, o professor terá que cadastrar um enunciado, qual linguagem é para ser utilizada na resolução, o nível (fácil, médio ou difícil), o tipo do exercício (condicional ou laço de repetição), a resposta que poderá ser escrita em um campo ou um \foreign{upload} do arquivo com as respostas. Cada exercício cadastrado será agregado a uma lista de exercícios que o professor irá criar antes de cadastrar os exercícios.


% TODO: Colocar formato de exemplo de uma resposta

\subsection{Cadastro de Turma}

O professor poderá realizar o cadastro de uma classe para adicionar seus alunos e passar as listas de exercícios e acompanhar o rendimento de cada aluno. Para cadastrar uma turma é necessário um nome da turma e um professor, com isso o professor poderá adicionar os alunos desejados.

\subsection{Realização de Exercícios}

Após a conclusão do cadastro de usuário, o aluno ou professor terão acesso as funcionalidades especificas de cada tipo de usuário. Dessa forma, o usuário poderá resolver uma lista de exercícios de duas formas: estando vinculado a uma turma ou individualmente. Caso o usuário esteja vinculado a uma turma, a lista de exercícios que ele poderá resolver será a que o professor responsável pela turma criou. Caso o usuário não faça parte de uma turma, ele terá que pesquisar por uma lista de exercícios informando qual assunto, estrutura de condição ou laço de repetição, ele deseja realizar. 

Depois que o usuário encontrar uma lista de exercícios, o sistema apresentará os exercícios à serem resolvidos na ordem em que foram adicionados na lista. Assim o usuário poderá trabalhar em sua solução e submeter ao sistema para correção utilizando a linguagem de programação escolhida pelo professor na criação da lista de exercícios, essa resolução poderá estar correta ou não. Se estiver correta, o sistema apresentará uma mensagem de parabenização e redirecionará o usuário para o próximo exercício. Caso a resolução estiver errada o sistema retornará qual o erro ocorrido na compilação, assim o usuário poderá realizar outras submissões até que consiga obter sucesso na resolução ou requisitar uma dica para o sistema para auxiliá-lo na construção da resposta. 

Cada submissão, tanto correta ou incorreta, é gravada no banco de dados na forma de \foreign{log}, sendo salvo os dados: a usuário que realizou o exercício, todas as submissões, os erros cometidos, o tempo demorado para obter sucesso no exercício, data e dicas utilizadas.

\subsubsection{Utilização das Dicas}

O usuário apenas poderá consultar uma dica caso o exercício enviado para validação não esteja correto e ele escolha a opção de disponibilização de dicas, assim ele terá direito a três dicas selecionadas aleatoriamente pelo mecanismo. O usuário irá escolher consultar uma das três dicas e após a consulta ele poderá realizar uma nova submissão do exercício.

\subsubsection{Boca (BOCA Online Contest Administrator)}

Para realizar a compilação dos exercícios no sistema de dicas, será utilizado funções do \foreign{software} BOCA \citeonline{de2004boca} que é escrito predominantemente em PHP, o que permitiria sua execução em qualquer plataforma compatível com tal linguagem. O Boca é amplamente utilizado em maratonas de programação em todo o Brasil, são exemplos de maratonas: Maratona de Programação da USP, Maratona de Programação da FACENS, Maratona de Programação da Faculdade de Informática de Presidente Prudente e Copa de programação da PUC-SP. 

\subsection{Cadastro de Dica}

O cadastro de uma dica pode ser realizado quando o usuário submete um exercício para validação e obtêm êxito, assim o sistema apresentará uma tela perguntando se o usuário deseja contribuir com uma dica, caso aceite, o sistema irá redirecionar o usuário para a tela onde será realizado o cadastro da dica, onde ele irá descrever em um campo a dica que deseja compartilhar. Também será apresentada a solução do mesmo exercício de outro usuário para que seja escrito uma dica. 

\subsection{Cadastro do Diário}

No processo de execução de um exercício o usuário poderá preencher um diário relatando suas experiências, essa funcionalidade será apresentada aos usuários caso o professor deseja receber o preenchimento do diário dos exercícios da lista. 

Caso o professor pretende receber os diários dos exercícios da lista de sua turma, o usuário após cada submissão da resolução no sistema produzirá um diário que será informado os dados referentes a experiência de realizar o exercício. 

\section{Formato do Estudo}

Esta seção apresentará o formato do estudo, descrevendo como serão realizados o teste do sistema, a criação do banco de exercícios e banco de dicas e como serão respondidas as questões de pesquisa.

\subsection{Teste do sistema}

Após o desenvolvimento do sistema ser concluído. Antes de realizar os estudos, será necessário executar uma etapa de teste para encontrar possíveis erros no sistema. Para que essa etapa aconteça, será con. Rese vocado um grupo de alunos de forma voluntária para utilizarem as funcionalidades do sistema e reportar se existe algum erro no sistema. Os alunos deverão reportar os erros encontrados através de criação de \foreign{issues} no \foreign{GitHub} onde está o repositório do sistema. Todos os erros serão corrigidos antes de realizar os estudos para avaliar o sistema.

\subsection{Banco de Exercícios}

O banco de dados de exercícios será criado a partir de exercícios selecionados para diferir dos exercícios realizados na matéria de Algoritmos oferecida pelo curso, a seleção terá a finalidade de prevenir que o voluntário do estudo de avaliação do sistema já tenha realizado o exercício. Os exercícios serão divididos em três grupos, o primeiro abordará conceitos básicos da linguagem de programação como entrada e saída de dados, declaração de variáveis e constantes, operações e funções matemáticas. O segundo grupo refere-se a estruturas condicionais e o terceiro grupo trata de estruturas de repetição.

\subsection{Banco de Dicas}

\begin{figure}[ht]
	\captionsetup{justification=centering}
	\includegraphics[width=\linewidth]{fluxo.png}
	\caption{Fluxos para criação de dicas.}
	\label{figura:dicafluxo}
\end{figure}


O banco de dicas será provido após a conclusão do banco de exercícios, assim será realizada uma convocação por formulário \textit{online} aos estudantes da Universidade Tecnológica Federal do Paraná do curso de Bacharelado em Ciência da Computação para se voluntariarem a utilizar o sistema de dicas por um determinado período em um dia que será estipulado no formulário. Os voluntários utilizaram o sistema de dicas em um ambiente controlado sendo esse um laboratório da Universidade Tecnológica Federal do Paraná com um professor ou alunos monitorando as atividades e tirando as possíveis dúvidas dos participantes. 

A criação das dicas conforme a Figura \ref{figura:dicafluxo} irá respeitar dois fluxos de execução, sendo eles: fluxo de reflexão e fluxo de comparação. Os dois fluxos são iniciados quando o usuário recebe uma lista de exercícios para ser realizada. Após o usuário ter uma lista de exercícios o fluxo de reflexão começa, o sistema irá apresentar um exercício da lista para ser resolvido e o usuário poderá começar a realizar a solução. Após a solução ser concluída, o usuário submete a solução para validação, o sistema irá executar a solução do exercício e verificar se está correta ou não. Caso a solução não estiver correta, o sistema redireciona para a etapa de realização de solução, nessa etapa o usuário poderá pedir uma dica para o sistema. Caso a solução estiver correta, o sistema irá redirecionar o usuário para a tela aonde ele criará a dica para o exercício e irá realizar o cadastro dela no banco de dicas.

Após o usuário gerar uma dica do exercício realizado, o sistema irá seguir o fluxo de comparação. Deste modo, o sistema irá procurar no banco de dados soluções de outros usuários do exercício realizado. O sistema irá apresentar ao usuário a pior e a melhor solução do exercício para que ele crie uma dica para melhorar as duas soluções. As dicas de otimização serão cadastradas no banco de dados de dicas para auxiliar outros usuários na resolução do exercício.

\subsection{Avaliação do Sistema de Dicas}

\begin{figure}[ht]
	\captionsetup{justification=centering}
	\includegraphics[width=\linewidth]{estudo.png}
	\caption{Visão geral da aplicação do estudo.}
	\label{figura:estudo}
\end{figure}

Nesta subseção será explicado as duas etapas que serão realizadas para responder as duas questões de pesquisa. A primeira etapa consiste em aplicar um estudo controlado na Universidade Tecnológica Federal do Paraná para gerar os dados de \foreign{log} de execução das listas de exercícios necessários. Dessa forma, após a geração dos dados de \foreign{log} a segunda etapa será inicializada. A segunda etapa apresenta as análises que serão feitas para responder as duas questões de pesquisa.

\begin{table}[]
	\centering
	\captionsetup{justification=centering}
	\caption{Formulário para aquisição de voluntários.}
	\label{tabela:formulário}
	\begin{tabular}{l}
		\hline
		Perguntas                        \\ \hline
		Nome?                            \\
		Idade?                           \\
		Telefone?                        \\
		Email?                           \\
		Curso?                           \\
		Tempo que está cursando o curso? \\
		Semestre?                        \\ \hline
	\end{tabular}
\end{table}

A Figura \ref{figura:estudo} representa a primeira etapa da avaliação do sistema, onde será necessário a colaboração de voluntários do curso de Bacharelado em Ciência da Computação para o experimento presencial que será realizado na Universidade Tecnológica Federal do Paraná, para isso será criado um formulário no \foreign{Google Forms} representado na Tabela \ref{tabela:formulário} e disponibilizado \foreign{online} para os alunos do curso.

Os voluntários serão divididos em três grupos distintos de acordo com o nível de conhecimento em programação, o nível de cada voluntário será estimado de acordo com o tempo que está na graduação e o semestre que se encontra. Os três grupos irão realizar os mesmos exercícios mas utilizaram funcionalidades diferentes do sistema,  

\begin{itemize}
	\item \textbf{Grupo 1}: utilizará o mecanismo de dicas que prove dicas de acordo com o exercício que o aluno está realizando.
	
	\item \textbf{Grupo 2}: utilizará o mecanismo de dicas personalizado que disponibiliza dicas de acordo com o exercício e o erro cometido na submissão.

	\item \textbf{Grupo 3}: utilizará o sistema sem o mecanismo de dicas.
\end{itemize}

Toda submissão realizada pelos três grupos de voluntários será gravada no \foreign{log} de submissões, sendo salvo os dados: a usuário que realizou o exercício, todas as submissões, os erros cometidos, o tempo demorado para obter sucesso no exercício. Após todos os voluntários tiverem finalizados todos os exercícios da lista, o estudo presencial será finalizado com a aplicação de um questionário representado na Tabela \ref{tabela:questionárioestudo} para avaliar a usabilidade do sistema e a experiência da utilização de um sistema \foreign{web} para solucionar exercícios de programação.

\begin{table}[]
	\centering
	\captionsetup{justification=centering}
	\caption{Questionário do estudo presencial.}
	\label{tabela:questionárioestudo}
	\begin{tabular}{l}
		\hline
		Perguntas                        \\ \hline
		pergunta1?                            \\
		pergunta2?                           \\
		pergunta3?                        \\
		pergunta4?                           \\
		pergunta5?                           \\
		pergunta6? \\
		pergunta7?                        \\ \hline
	\end{tabular}
\end{table}

A segunda etapa da avaliação do sistema responderá as duas questões de pesquisa, será analisado os dados do \foreign{log} de submissões realizados na primeira etapa. A primeira questão de pesquisa tem como objetivo investigar se o mecanismo de dicas irá ajudar os alunos a obterem melhores resultados na execução de exercícios de programação, essa questão será respondida a partir da avaliação do desempenho dos grupos que utilizaram o sistema com o mecanismo de dicas em relação ao grupo que utilizou o sistema sem o mecanismo, também será avaliado o questionário respondido pelos voluntários.   

A segunda questão: Quais dicas personalizadas ajudam mais os alunos? Será respondida com a análise da comparação do desempenho do primeiro grupo de voluntário com o segundo, e verificando se as dicas de voluntários mais experientes possuíram melhor avaliação em relação as dicas dos voluntários menos experientes.