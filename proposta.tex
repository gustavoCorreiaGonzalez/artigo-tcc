\chapter{Proposta}

Este capítulo apresentará o método utilizado para a elaboração deste trabalho. Assim, para avaliar o impacto do uso do mecanismo de dicas personalizadas no ensino de conceitos básicos de programação, primeiramente será realizado a implementação do \foreign{iHint} utilizando o \foreign{framework} de desenvolvimento Laravel\footnote{\url{https://laravel.com/}}. Subsequente, será efetuado um estudo com os dados gerados nos experimentos presenciais para realizar a avaliação do sistema. A \cref{figura:visaometodo} ilustra cada etapa a ser realizada para o desenvolvimento da abordagem proposta.

\begin{figure}[ht]
	\captionsetup{justification=centering}
	\includegraphics[width=\linewidth]{imagens/visaogeraldometodoproposto.png}
	\caption{Visão geral do método proposto.}
	\label{figura:visaometodo}
\end{figure}

\section{Teste do Sistema}

Após o desenvolvimento do sistema ser concluído, será necessário executar uma etapa de teste para encontrar possíveis erros no sistema. Para que essa etapa aconteça, será convocado um grupo de alunos da UTFPR do Curso de Bacharelado em Ciência da Computação (CBCC) de forma voluntária para utilizarem as funcionalidades do sistema, o estudo será realizado em um laboratório com a supervisão de um professor ou aluno. Os alunos deverão reportar os erros encontrados através de criação de \foreign{issues} no \foreign{GitHub} \footnote{\url{https://github.com/gustavoCorreiaGonzalez/aplicacao-tcc }}. Todos os erros serão corrigidos antes de realizar os estudos para avaliar o sistema.

\section{Estudo de Avaliação}

Esta seção apresentará o formato do estudo, descrevendo como serão realizados o teste do sistema, a criação do banco de exercícios, banco de dicas e como serão respondidas as questões de pesquisa.

\subsection{Banco de Exercícios}

Para que o estudo e a avaliação do sistema sejam possíveis, será necessário criar um banco de dados de exercícios a partir de exercícios selecionados para diferir dos exercícios realizados na matéria de Algoritmos oferecida pelo curso, a seleção terá a finalidade de prevenir que o voluntário do estudo de avaliação do sistema já tenha realizado o exercício. Os exercícios serão divididos em dois grupos, o primeiro abordará estruturas condicionais e o segundo grupo trata de estruturas de repetição.

\subsection{Banco de Dicas}

Para realizar o estudo UTFPR com os voluntários, será preciso que o sistema já tenha algumas dicas cadastradas no banco de dados. Deste modo, o banco de dicas será provido após a conclusão do banco de exercícios, assim será realizada uma convocação por formulário \foreign{online} aos estudantes da UTFPR do CBCC para se voluntariarem a utilizar o sistema de dicas por um determinado período em um dia que será estipulado no formulário. Os voluntários utilizaram o sistema de dicas em um ambiente controlado sendo esse um laboratório da UTFPR com um professor ou aluno monitorando as atividades e tirando as possíveis dúvidas dos participantes. 

\begin{figure}[ht]
	\centering
	\captionsetup{justification=centering}
	\includegraphics[width=.5\linewidth]{imagens/fluxoreflexao.png}
	\caption[Fluxo de reflexão para criação de dicas]{Fluxo de reflexão para criação de dicas. \\ Fonte: adaptada de \citeonline{Glassman:2016:LPH:2818048.2820011}}
	\label{figura:fluxoreflexao}
\end{figure}

\begin{figure}[ht]
	\centering
	\captionsetup{justification=centering}
	\includegraphics[width=.5\linewidth]{imagens/fluxocomparacao.png}
	\caption[Fluxo de comparação para criação de dicas]{Fluxo de comparação para criação de dicas. \\ Fonte: adaptada de \citeonline{Glassman:2016:LPH:2818048.2820011}}
	\label{figura:fluxocomparacao}
\end{figure}

A criação das dicas ocorrerá conforme dois fluxos de execução, sendo eles: fluxo de reflexão representado pela \cref{figura:fluxoreflexao} e fluxo de comparação representado pela \cref{figura:fluxocomparacao}. Os dois fluxos são iniciados quando o usuário recebe uma lista de exercícios para ser realizada. No momento em que o usuário ter uma lista de exercícios o fluxo de reflexão começa. O sistema irá apresentar um exercício da lista para ser resolvido e o usuário poderá começar a realizar a solução, o aluno poderá requisitar ao sistema que necessita realizar uma consulta a algumas dicas. Deste modo, o sistema apresentará 5 dicas para o aluno consultar e realizar sua solução.

Após a solução ser concluída, o usuário submete a solução para validação, o BOCA que irá executar a solução do exercício e verificar se está correta ou não. Caso a solução não estiver correta, o sistema redireciona para a etapa de realização de solução, nessa etapa o usuário poderá pedir 5 dicas para o sistema. Caso a solução estiver correta, o sistema irá redirecionar o usuário para a tela aonde ele criará a dica para o exercício e irá realizar o cadastro dela no banco de dicas.

Após o usuário gerar uma dica do exercício realizado, o sistema irá seguir o fluxo de comparação. Neste fluxo, o sistema irá procurar no banco de dados soluções de outros usuários do exercício realizado. O sistema irá apresentar ao usuário uma das piores e a uma das melhores soluções do exercício realizadas por outros usuários para que ele crie uma dica para melhorar as soluções. Para o sistema conseguir distinguir uma boa solução de uma ruim, cada solução será classificada tendo em conta os dados da execução do exercício e as métricas retiradas do código. Na execução do exercício será avaliado o tempo para a resolução correta do exercício, quantidade de dicas utilizadas e quantidades de tentativas erradas. No entanto, as métricas do código serão medidas através do número de linhas entre as tentativas, complexidade ciclomática e tamanho da solução. Com esses dados analisados, o sistema poderá realizar um \foreign{ranking} com as soluções e será considerado 25\% das primeiras posições como o grupo das melhores soluções e 25\% das últimas colocações como sendo o grupo das piores soluções. Assim, o sistema irá retornar para o aluno duas soluções que serão selecionadas de forma aleatória, sendo uma do grupo das melhores soluções e a outra do grupo das piores soluções.

A partir das soluções apresentadas ao aluno pelo sistema, ele poderá analisar e gerar uma dica para melhorar as soluções. Essas dicas de otimização serão cadastradas no banco de dados de dicas para auxiliar outros usuários na resolução do exercício.

\subsection{Avaliação do Sistema de Dicas}

Nesta subseção será explicado as duas etapas que serão realizadas para responder as duas questões de pesquisa citadas abaixo.

\begin{itemize}
	\item \textbf{QP}\textsubscript{1}: 
	\textit{Existe impacto do uso do mecanismo de dicas para minimizar erros e melhorar soluções de exercícios?}
	\foreign
	\item \textbf{QP}\textsubscript{2}: 
	\textit{Quais dicas personalizadas ajudam mais os alunos?}
\end{itemize}

Enquanto \textbf{QP}\textsubscript{1} tem como objetivo investigar se o mecanismo de dicas irá ajudar os alunos a obterem melhores resultados na resolução de exercícios de programação. \textbf{QP}\textsubscript{2} avaliará qual tipo de dica é melhor, podendo ser aleatorizadas em função do tipo do exercício ou geradas a partir de erros cometidos em exercícios passados. Levando em consideração a fonte da dica gerada, podendo ser por alunos iniciantes, alunos experientes ou professores.

\begin{figure}[ht]
	\centering
	\captionsetup{justification=centering}
	\includegraphics[width=.6\linewidth]{imagens/estudo.png}
	\caption{Visão geral da aplicação do estudo.}
	\label{figura:estudo}
\end{figure}

A primeira etapa consiste em aplicar um estudo controlado na UTFPR para gerar os dados de \foreign{log} de submissões das listas de exercícios. Dessa forma, após a geração dos dados de \foreign{log} a segunda etapa será inicializada. A segunda etapa apresenta as análises que serão feitas para responder as questões de pesquisa.

\begin{table}[ht]
	\centering
	\captionsetup{justification=centering}
	\caption{Formulário para aquisição de voluntários.}
	\label{tabela:formulário}
	\begin{tabular}{l}
		\hline
		Perguntas                        \\ \hline
		Nome?                            \\
		Idade?                           \\
		Telefone?                        \\
		E-mail?                          \\
		Curso?                           \\
		Tempo que está cursando o curso? \\
		Qual semestre você está?         \\ 
		Quantidade de vezes que cursou a disciplina de Algoritmos?  \\ 
		Teve experiência com programação antes da graduação? \\
		Trabalha ou já trabalhou com programação? \\ \hline
	\end{tabular}
\end{table}

A Figura \ref{figura:estudo} representa a primeira etapa da avaliação do sistema, onde será necessário a colaboração de voluntários do CBCC para o experimento presencial que será realizado na UTFPR, para isso será criado um formulário no \foreign{Google Forms} representado na Tabela \ref{tabela:formulário} e disponibilizado \foreign{online} para os alunos do curso.

Os voluntários serão divididos em três grupos distintos de acordo com o nível de conhecimento em programação, o nível de cada voluntário será estimado de acordo com o tempo que está na graduação e o semestre que se encontra. Os três grupos irão realizar os mesmos exercícios mas utilizaram o sistema de três formas diferentes:

\begin{itemize}
	\item \textbf{Grupo 1}: Utilizará o mecanismo de dicas que prove dicas de acordo com o exercício que o aluno está realizando.
	
	\item \textbf{Grupo 2}: Utilizará o mecanismo de dicas personalizado que disponibiliza dicas de acordo com o exercício e o erro cometido na submissão.
	
	\item \textbf{Grupo 3}: Utilizará o sistema sem o mecanismo de dicas.
\end{itemize}

Toda submissão realizada pelos três grupos de voluntários será gravada no \foreign{log} de submissões, sendo salvo os dados: o usuário que realizou o exercício, todas as submissões, os erros cometidos, o tempo demorado para obter sucesso no exercício. Após todos os voluntários tiverem finalizados todos os exercícios da lista, o estudo presencial será finalizado com a aplicação de um questionário para os voluntários que utilizaram o mecanismo de dicas representado na \cref{tabela:questionárioestudocomdica} e outro questionário para os voluntários que não utilizaram o mecanismo de dicas representado na \cref{tabela:questionárioestudosemdica} para avaliar a usabilidade do sistema e a experiência da utilização de um sistema \foreign{web} para solucionar exercícios de programação. As respostas dos questionários será realizada na escala \citeonline{likert1932technique} de 5 pontos (1: Não concordo totalmente, 2: Não concordo parcialmente, 3: Indiferente, 4: Concordo parcialmente, 5: Concordo totalmente).

\begin{table}[ht]
	\centering
	\captionsetup{justification=centering}
	\caption{Questionário do estudo presencial para os voluntários que utilizaram o mecanismo de dicas.}
	\label{tabela:questionárioestudocomdica}
	\begin{tabular}{l}
		\hline
		Questionário                        											\\ \hline
		Foi fácil utilizar o \foreign{iHint}?                            				\\
		Precisou do apoio de uma pessoa para usar o sistema?                           	\\
		A interface do sistema é agradável?                        						\\
		O sistema oferece todas as informações necessárias para completar as tarefas?   \\
		Você recomendaria este sistema para outras pessoas?                           	\\
		O objetivo do sistema foi alcançado? 											\\ 
		O mecanismo de dica ajudou na resolução dos exercícios?							\\
		Foi fácil acessar as dicas pelo sistema?										\\
		As dicas foram apresentadas de forma clara e objetiva							\\ \hline
	\end{tabular}
\end{table}

\begin{table}[ht]
	\centering
	\captionsetup{justification=centering}
	\caption{Questionário do estudo presencial para os voluntários que não utilizaram o mecanismo de dicas.}
	\label{tabela:questionárioestudosemdica}
	\begin{tabular}{l}
		\hline
		Questionário                        											\\ \hline
		Foi fácil utilizar o \foreign{iHint}?                            				\\
		Precisou do apoio de uma pessoa para usar o sistema?                           	\\
		A interface do sistema é agradável?                        						\\
		O sistema oferece todas as informações necessárias para completar as tarefas?   \\
		Você recomendaria este sistema para outras pessoas?                           	\\
		O objetivo do sistema foi alcançado? 											\\ \hline
	\end{tabular}
\end{table}

A segunda etapa da avaliação do sistema responderá as duas questões de pesquisa, será analisado os dados do \foreign{log} de submissões realizados na primeira etapa. A primeira questão de pesquisa será respondida a partir da avaliação do desempenho dos grupos que utilizaram o sistema no estudo controlado. Será realizada uma comparação estatística para avaliar as hipóteses descritas abaixo. 

\begin{itemize}
	\item \textbf{Hipótese 1}: O grupo 1 desenvolverá soluções melhores com a utilização dos mecanismos de dicas em relação as soluções do grupo 3.
	\item \textbf{Hipótese 2}: O grupo 2 desenvolverá soluções melhores com a utilização dos mecanismos de dicas personalizadas em relação as soluções do grupo 3.	
	\item \textbf{Hipótese 3}: O grupo 1 desenvolverá soluções melhores com a utilização dos mecanismos de dicas em relação as soluções do grupo 2.	
	\item \textbf{Hipótese 4}: O grupo 2 desenvolverá soluções melhores com a utilização dos mecanismos de dicas personalizadas em relação as soluções do grupo 1.
\end{itemize}

Utilizaremos o método estatístico \foreign{Mann-Whitney-Wilcoxon Test}\footnote{\url{http://www.r-tutor.com/elementary-statistics/non-parametric-methods/mann-whitney-wilcoxon-test}} para analisar as métricas: número de linhas entre as tentativas, complexidade ciclomática e tamanho da solução. Como cada grupo de voluntários contém alunos diferentes, então os dados gerados de cada grupo não afetaram uns aos outros, portanto são independentes. Assim, sem assumir que os dados tem distribuição normal, decidimos em 0,05 o nível de significância se os dados do número de linhas entre as tentativas, complexidade ciclomática e tamanho da solução tem distribuição de dados idênticos. 

Deste modo, a hipótese nula é quando os valores que são parte da amostra de cada grupo vem de populações idênticas. Para testar as hipóteses, será aplicado o teste estatístico para comparar as amostras e analisar o valor de \foreign{p} que será retornado, este valor será comparado com o nível de significância e caso o valor de \foreign{p} seja menor, rejeitamos a hipótese nula, tornando assim a hipótese verdadeira. Caso o valor de \foreign{p} seja maior, aceitamos a hipótese nula e a hipótese será falsa. 

Também será aplicado o teste \foreign{Effect Size} \cite{macbeth2011cliff}, para verificar a significância dos dados gerados por cada grupo de voluntários, será verificado se os dados possuem eventuais diferenças entre as médias ou variâncias. Mas como o estudo presencial não foi realizado, não temos os dados para análise e não foi possível escolher qual método de média ou variância será utilizado.

A segunda questão será respondida com a análise estatística da comparação do desempenho do grupo 1 de voluntário com o grupo 2, será verificado se as dicas dos voluntários mais experientes possuíram melhor avaliação em relação as dicas dos voluntários menos experientes. Os usuários serão divididos em mais experiente e menos experiente através dos dados informados no cadastro, sendo os dados cadastrais: o tempo que está cursando o CBCC, o semestre que está cursando, a quantidade de vezes que cursou a disciplina de Algoritmos, se o usuário teve alguma experiência com programação antes da graduação, se trabalha ou já trabalhou com programação e se é um aluno ou professor. Assim, a partir desses dados será possível classificar os usuários de acordo com sua experiência na faculdade e fora dela. Após ser realizada a classificação dos usuários, será classificada as dicas a partir da avaliação feita pelos voluntários no estudo presencial, assim será analisado que tipo de usuário contribuiu para as dicas que mais ajudaram os voluntários no estudo presencial.
