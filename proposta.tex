\chapter{Proposta}

Neste capítulo apresenta o método utilizado para a elaboração deste trabalho. A Figura \ref{figura:visaometodo} ilustra cada etapa a ser realizada para o desenvolvimento da abordagem proposta.

\begin{figure}[h]
	\captionsetup{justification=centering}
	\includegraphics[width=\linewidth]{visaogeraldometodoproposto.png}
	\caption{Visão geral do método proposto.}
	\label{figura:visaometodo}
\end{figure}

Para avaliar o impacto do uso do mecanismo de dicas personalizadas no ensino de conceitos básicos de programação, primeiramente será realizado a implementação do sistema de dicas utilizando o \foreign{framework} de desenvolvimento Laravel na versão 5.1. 

%Será realizado a implementação do sistema de dicas utilizando o \foreign{framework} de desenvolvimento Laravel na versão 5.1. A partir do sistema desenvolvido, será realizado a construção da base de dados de exercícios do qual constituirá de exercícios referentes a estrutura condicional e laço de repetição. Na etapa seguinte a partir do banco de dados de exercícios formaremos o bando de dados de dicas, para realizar essa etapa será necessário a colaboração de alguns voluntários para utilizar o sistema de dicas e realizar os exercícios e fornecer dicas para o banco de dicas. Em seguida, será realizado um teste do sistema antes da realização do estudo prártico. Por fim, será realizado a etapa de validação que constituirá de três estudos isolados com volutários que utilização o sistema de dicas. A seguir, todas as etapas serão descritas em suas respectivas seções.

\section{Implementação do Mecanismo de Dicas}

Essa seção, descreve o sistema de dicas a ser desenvolvido. O processo de realização de exercícios, que é onde o sistema deve atuar, pode ser dividi do em quatro fases: cadastro do usuário, realização de exercícios, utilização de dicas (providas por usuários do sistema) e validação do exercício. A descrição apresentada será usada para a modelagem do sistema.

\begin{figure}[h]
	\captionsetup{justification=centering}
	\includegraphics[width=\linewidth]{sistemadicas.png}
	\caption{Visão geral do sistema de dicas.}
	\label{figura:sistemadicas}
\end{figure}

\subsection{Cadastro do Usuário}

Quando um usuário desejar realizar o cadastro no nosso sistema, ele poderá escolher entre a opção de aluno ou professor. O aluno, realiza exercícios disponíveis no banco de dados do sistema, preenche um diário não obrigatório para cada exercício, cria uma dica para cada exercício e consulta seu perfil com seus dados pessoais e um relatório de suas submissões de exercícios. O professor, além de realizar as mesmas atividades do aluno, também poderá criar salas com exercícios pré-definidos convidando alunos para fazer parte dela e submeter exercícios para o sistema.

\subsection{Cadastro de Exercícios}

Essa funcionalidade do sistema só poderá ser acessada se o usuário for um professor. Sendo assim, o usuário terá que cadastrar um enunciado, qual linguagem é para ser utilizada na resolução, o nivel (fácil, médio ou difícil), o tipo do exercício (condicional, laço de repetição, função, etc), a resposta que poderá ser escrita em um campo ou um \foreign{upload} do arquivo com as respostas. O professor poderá opinar por adicionar o exercício cadastrado em uma lista de exercícios. 

% TODO: Colocar formato de exemplo de uma resposta
\subsection{Cadastro de Turma}

O professor poderá realizar o cadastro de uma classe para adicionar seus alunos e passar as listas de exercícios e acompanhar o rendimento de cada aluno. Para cadastrar uma turma é necessário um nome da turma e um professor, com isso o professor poderá adicionar os alunos desejados.


% TODO: Olhei até aqui


\subsection{Realização de Exercícios}

Após a conclusão do cadastro de usuário, o aluno ou professor é encaminhado para a tela principal do sistema onde poderá realizar as atividades permitidas para cada tipo de usuário. Caso a opção seja realizar um exercício, o usuário poderá escolher qual tipo e nível de dificuldade do exercício deseja realizar, após preencher esses dados o sistema retornará uma lista com alguns exercícios que satisfazem os campos de busca. O usuário irá escolher um exercício, realiza-lo na linguagem de programação que deseja e submeter para a validação do sistema. 

O exercício poderá estar correto ou não. Se estiver correto o sistema irá apresentar uma mensagem de parabenização e irá redirecionar o usuário para página de busca de exercícios, caso seja rejeitado o sistema retornará qual tipo de erro ocorreu na compilação do exercício, o usuário poderá realizar submissões até que consiga obter sucesso na submissão ou requisitar uma dica para o sistema. Cada submissão, tanto correta ou incorreta, é gravada no banco de dados na forma de \foreign{log}, sendo salvo o usuário que realizou o exercício, todas as submissões, os erros cometidos, o tempo demorado para obter sucesso no exercício.

\subsubsection{Utilização das Dicas}

O usuário apenas poderá consultar uma dica caso o exercício enviado para validação não estaja correto e ele escolha a opção de disponibilização de dicas, assim ele terá direito a três dicas selecionadas aleatoriamente pelo mecanismo. O usuário irá escolher consultar uma das três dicas e após a consulta ele poderá realizar uma nova submissão do exercício.

\subsection{Cadastro de Dica}

O cadastro de uma dica pode ser realizado quando o usuário submete um exercício para validação e obtêm exito, assim o sistema apresentará uma tela perguntando se o usuário deseja contribuir com uma dica, caso aceite, o sistema irá redirecionar o usuário para a tela onde será realizado o cadastro da dica, onde ele irá descrever em um campo a dica que deseja compartilhar. Também será apresentada a solução do mesmo exercício de outro usuário para que seja escrito uma dica. 

\subsection{Cadastro do Diário}

No processo de execução de um exercício o usuário poderá preencher um diário relatando suas experiências, para ocorrer essa atividade o usuário irá precisar acionar a opção de preenchimento do diário.  

Após o usuário ter realizado um exercício e obter sucesso, o sistema perguntará se o usuário deseja realizar a escrita de um diáro para o respectivo exercício, caso aceite, o sistema redirecionará o usuário para a página do diário onde o usuário informará os dados referentes as submissões dos exercícios.

\subsection{Boca}

Para realizar a compilação dos exercícios no sistema de dicas, será utilizado funções do \foreign{software} BOCA que é escrito predominantemente em PHP, o que permitiria sua execução em qualquer plataforma compatível com tal linguagem. No entanto, o sistema de julgamento depende de scripts Bash e de funcionalidades específicas do sistema operacional Linux (isolamento de recursos/jail). 

\section{Banco de Exercícios}

O banco de dados de exercícios será criado a partir de exercícios selecionados para diferir dos exercícios realizados na matéria de Algoritmos oferecida pelo curso, a seleção terá a finalidade de previnir que o voluntário do estudo de validação do sistema já tenha realizado o exercício. Os exercícios seram divididos em três grupos, o primeiro abordará conceitos básicos da linguagem de programação como entrada e saída de dados, declaração de varíavies e constantes, operações e funções matemáticas. O segundo grupo refere-se a estruturas condicionais e o terceiro grupo trata de estruturas de repetição.

\section{Banco de Dicas}

O banco de dicas será provido após a conclusão do banco de exercícios, assim realizaremos uma convocação por formulário \textit{online} aos estudantes da Universidade Tecnológica Federal do Paraná do curso de Bacharelado em Ciência da Computação para se voluntariarem a utilizar o sistema de dicas por um determinado período em um dia que será estipulado no formulário. Os voluntários utilizaram o sistema de dicas em um ambiente controlado sendo esse um laboratório da Universidade Tecnológica Federal do Paraná com um professor ou alunos monitorando as atividades e tirando as possíveis dúvidas dos participantes. Cada participante terá que realizar todos os exercícios cadastrados na base de dados e cadastrar dicas para cada um.