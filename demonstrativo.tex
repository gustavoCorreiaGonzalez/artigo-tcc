\chapter{Demonstrativo da aplicação dos recursos}

Para realizar o estudo, o sistema deve estar hospedado em um servidor para que todos os voluntários tenham acesso em qualquer computador com Internet. Esse servidor necessita de bom desempenho para atender a demanda de acessos que serão realizados durante os experimentos. Para hospedar o sistema de dicas, foi escolhido o Digital Ocean que provê o serviço de \foreign{Cloud computing}. Inicialmente, para o teste do sistema, foi contratado o serviço básico que disponibiliza um servidor, com o custo de US\$ 5 (cinco dólares) por mês, com as seguintes configurações:

\begin{itemize}
	\item 512MB de memória RAM
	\item 1 Core de processador
	\item 20GB de armazenamento de disco em SSD
	\item 1TB de transferência de rede
\end{itemize}

Embora essa configuração seja satisfatória para o desenvolvimento inicial do sistema, para a execução
dos estudos experimentais planejados, faz-se necessário contratar um plano com mais recursos, em
especial quanto à memória RAM e desempenho de processamento. No caso do serviço do Digital Ocean, o
plano mais adequado . O segundo plano, custando US\$ 40 (quarent dólares) por mês, oferece as seguintes configurações, julgadas adequadas para os propósitos do trabalho:

\begin{itemize}
	\item 4 GB de memória RAM,
	\item 2 núcleos de processamento,
	\item 60 GB de armazenamento de disco em SSD,
	\item 4 TB de transferência de rede.
\end{itemize}





\chapter{Demonstrativo de vinculação do projeto}

O estudo está inserido na área de Engenharia de Software e de Sistemas Colaborativos,
aplicado ao domínio de ensino em Computação. O \textit{software} resultante deste 
trabalho tem aplicação no domínio de ensino em Computação e os princípios são
extensíveis a outras área. Além disto, quanto aos aspectos tecnológicos, a aplicação
é adequada aos requisitos de aplicações Web esperadas para egressos em Computação.