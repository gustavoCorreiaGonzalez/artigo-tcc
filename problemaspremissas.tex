\chapter{Problemas e premissas}

O problema do insucesso é evidenciado por \citeonline{lahtinen2005study}, que realizaram uma pesquisa com diferentes universidades para estudar as dificuldades na aprendizagem de programação. Como resultado, foi percebido que as questões mais difíceis na programação são: a compreensão de como projetar um programa para resolver uma tarefa determinada; dividir as funcionalidades em procedimentos; e encontrar erros de seus próprios programas. Portanto, são essas as capacidades que os alunos devem obter para entender as maiores entidades do programa em vez de apenas alguns detalhes sobre eles.

Estas dificuldades contribuem para a desistência dos alunos nas disciplinas de programação, por isso os pesquisadores estão utilizando várias metodologias e \foreign{softwares} para minimizar esse problema.