\chapter{Referencial Teórico}

Neste capítulo apresentaremos uma revisão da literatura agrupando trabalhos relacionados a reprovações e dificuldades de alunos em matérias de programação, novas abordagens de ensino, sistemas de dicas. 

\section{Reprovações em matérias de programação}

O entendimento e aplicação dos conceitos estudados em disciplinas de programação é fundamental para que o aluno consiga desenvolver programas mais complexo. \citeonline{Helminen:2010:JPV:1879211.1879234} afirmam que a programação é uma competência essencial no curso de Ciência da Computação. Segundo \citeonline{bosse2015reprovaccoes} os estudantes normalmente apresentam uma grande dificuldade com os conteúdos abordados, ocasionando em reprovação ou desistência.

\citeonline{Sinclair:2015:MSE:2729094.2742586} agruparam pesquisas internacionaris como a \textit{National Survey of Student Engagement} (NSSE) que é entregue anualmente na América do Norte e Canadá durante os últimos 15 anos, o governo australiano realiza a pesquisa \textit{Nacional Universidade Experience} (UES) que é usada desde 2012 e no Reino Unido utiliza a pesquisa \textit{Survey Student Participation} (SES) que realizam a medida dos esforços dos alunos em atividades associadas com a aprendizagem efetiva. Esse estudo reúne dados relacionadas à Ciência da Computação sendo que os resultados desta meta-análise indicam que o curso de Ciência da Computação apresentam taxas mais baixas do que a média em muitos dos principais pontos de referência de engajamento.

\section{Sistemas de dicas}

\citeonline{Price:2015:IIF:2787622.2787748} está utilizando um sistema de tutores inteligentes (STI) que podem manter os alunos no caminho certo, na ausência de instrutores, fornecendo sugestões e advertências para os alunos que precisam de ajuda. Além disso, as técnicas baseadas em dados pode gerar esse feedback automaticamente a partir de tentativas dos estudantes anteriores de um problema. O autor está aplicando o seu estudo permitindo que os alunos escrevam códigos que se conectam com seus interesses, tais como jogos, aplicações e histórias. Como exemplo concreto, imagine um aluno que esteja desenvolvendo um jogo simple, quer requer a utilização de variáveis, laços de repetição, condicionais. Mas o aluno está com dificuldades em relação a uma funcionalidades que o jogo terá, ele poderá pedir uma dica para implementar essa funcionalidade que será gerada automaticamente pelo sistema com relação as tentativas anteriores de outros alunos.

\citeonline{Elkherj:2014:SSR:2556325.2567864} apontam que as sugestões utilizadas em sistemas de aprendizagem on-line para ajudar os alunos quando eles estão tendo dificuldades são fixados antes do tempo e não dependem das tentativas mal sucedidas que o aluno já fez. Isto limita severamente a eficácia das sugestões. Eles desenvolveram um sistema alternativo para dar dicas aos estudantes. A principal diferença é que o sistema permite um instrutor enviar uma dica para um estudante após o aluno ter feito várias tentativas para resolver o problema e falhou. Depois de analisar os erros do aluno, o instrutor é mais capaz de entender o problema no pensamento do aluno e enviar-lhes uma dica mais útil. O sistema foi implantado em um curso de probabilidade e estatística com 176 alunos, obtendo \textit{feedback} dos alunos muito positivo. Mas o desafio que os autores enfrentam é como escalar efetivamente o sistema de forma que todos os alunos que precisam de ajuda obtenham dicas eficazes. Pois com uma grande quantidade de alunos ficaria exaustivo para os instrutores avaliarem cada submissão dos exercícios de cada aluno para retornarem uma dica especifica do erro cometido. Então eles criaram um banco de dados de dica que permite que os instrutores reutilizem, compartilhem e melhorem em cima de sugestões escritas anteriormente.

\citeonline{Cummins:2016:IUH:2876034.2893379} investigaram  o uso de dicas para 4.652 usuários qualificados em um ambiente de aprendizagem on-line de grande escala chamado Isaac, que permite aos usuários para responder a perguntas de física com até cinco dicas. Foi investigado o comportamento do usuário ao usar dicas, engajamento dos usuários com desvanecimento (o processo de tornar-se gradualmente menos dependentes das dicas fornecidas), e estratégias de dicas incluindo decomposição, correção, verificação ou comparação. Como resultados obtidos, os alunos apresentaram estratégias para as resoluções dos exercícios sendo a mais comum é ver o conceito da dica para realizar a decomposição do problema e, em seguida, enviar uma resposta correta. A outra estratégia é usar os conceitos da dica para determinar se a pergunta pode ser respondida, uma grande proporção dos usuários que utilizaram essa estratégia acabaram não tentando responder a pergunta


- fazer um resumo das abordagens de ensino e o engajamento dos estudantes

\section{Abrodagens de ensino gerazaooo}


criar um capitulo de sumarização dos artigos relacionados e comentar o porque de usar o sistema de dicas que vou criar.