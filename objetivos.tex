\chapter{Objetivos}

O objetivo desse trabalho é criar um software de código aberto para auxiliar na aprendizagem de conceitos básicos de programação, implementando um sistema colaborativo de dicas escritas pelos próprios usuários. Para investigar se o uso do mecanismo de dicas personalizadas é capaz de melhorar o rendimento dos alunos nos exercícios de estrutura de condição e laço de repetição, será aplicado um experimento controlado com três grupos distintos de alunos. O primeiro grupo utilizará o mecanismo de dicas que consiste em prover dicas de acordo com o exercício que o aluno está realizando. O segundo grupo utilizará o mecanismo de dicas personalizado que disponibiliza dicas de acordo com o exercício e o erro cometido na submissão. Por fim, o terceiro grupo utilizará o sistema sem o mecanismo de dicas.
	
No software serão implementadas funcionalidades que permitam ao usuário resolver exercícios nas linguagens de programações C, C++ e Java, além de fornecer dicas para a solução de exercícios e um diário para relatar as dificuldades enfrentadas durante a resolução.
	
Para realizar a validação da ferramenta será realizado um estudo controlado na UTFPR com três grupos de estudantes escolhidos aleatoriamente de forma homogênea em relação ao nível de conhecimento em programação. Com o experimento será possível avaliar se o grupo de alunos com dicas tem melhor desempenho que alunos sem o suporte do mecanismo. O desempenho será medido por: tempo, número de tentativas até a solução correta, qualidade do código gerado medida através da complexidade ciclomática do código, número de linhas entre as tentativas e tamanho da solução. Por fim, será avaliado se a qualidade das dicas está diretamente relacionada com o nível de conhecimento do aluno e se as dicas de alunos experientes são melhores ou não em relação as dicas dos alunos com menos experiência.